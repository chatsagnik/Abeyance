\documentclass{article}
\usepackage[utf8]{inputenc}
\usepackage{amsmath}
\usepackage{etoolbox}

\title{Abeyance}
\author{Sagnik Chatterjee}
\date{May 2017}

\begin{document}

\maketitle

\section{Introduction}

\textbf{Problem Statement: Find the Minimum Average Latency of a Non-Linear Pipeline}\\
A Pipeline is a modified queue designed to implement instruction-level parallelism in a CPU. Using the concept of Pipelining we can overlap multiple instructions in execution.\\
The computer pipeline is divided in a number of stages. Each stage completes a part of an instruction in parallel. The stages are connected to each other to form a pipe - instructions enter at one end, progress through the stages, and exit at the other end.\\
Before we proceed, a few important things to note about pipelines are:
\begin{itemize}
    \item Pipelining does not decrease latency or the time for individual instruction execution.
    \item Pipelining increases instruction throughput. The throughput of the instruction pipeline is determined by how often an instruction exits the pipeline.
	\item Pipelining implements a form of parallelism called instruction-level parallelism within a single processor.
\end{itemize}

\subsection{Non-Linear Pipelining}

Unlike traditional pipelines (or linear pipelines), \textbf{dynamic} or non-linear pipelines allow for feedback and feed-forward connections between the various stages of a pipeline.\\
A couple of key differences between linear pipelines and non-linear pipelines are:
\begin{itemize}
    \item The Output of a Dynamic Pipeline is not necessarily produced from the last stage.
    \item Non-Linear Pipelines can be reconfigured to perform different actions at different times.
\end{itemize}
\section{Design}

The main crux of the problem is: Given a reservation table of a pipeline, find the minimum average latency for the pipeline.\\
\subsection{Input} 
A 2D array consisting of 0's and 1's. The rows represent the pipeline stages, and the columns represent the clock cycles. 1's in the 2D array represent the clock cycle at which an instruction was initiated in a particular pipeline stage.
\subsection{Output}
Our program produces the following output on being provided with a reservation table:
\begin{itemize}
    \item A single floating point value denoting the minimum average latency of the given pipeline.
    \item A list of all possible average latencies in the given pipeline.
    \item A description of all the states in the state diagram. This includes a detailed description of all the vertices and the edges present in the state diagram.
    \item A Text File containing all the elementary cycles possible in the state diagram. An \textbf{Elementary Cycle} is one in which no vertex is repeated twice except for the start node.
\end{itemize}

\subsection{Procedure}:
The whole process can be divided into 8 atomic stages which must be conducted sequentially for us to derive the aforementioned outputs from the reservation table. These stages are listed below:
\begin{enumerate}
    \item Find non-repeating Forbidden Latencies from reservation table.
    \item Sort this list in ascending order and form the initial collision vector.
    \item Start constructing the state diagram from the initial collision vector.
    \item Convert the state diagram into a directed graph.
    \item Find all elementary cycles in the aforementioned directed graph.
    \item Find the average latency for each such mentioned elementary cycle.
    \item Select all elementary cycles with average latency which is less than the number of forbidden latencies.
    \item Select the smallest such latency.
\end{enumerate}

\section{Main Design Issues}
There were four main design issues in the whole program, which are listed as follows:
\begin{enumerate}
    \item Designing the structure which would be used for representing the different states of the state diagram.
    \item Actually populating the state diagram.
    \item Converting the state diagram into a directed graph.
    \item Finding all Elementary Circuits of the directed graph.
\end{enumerate}

\subsection{Designing the States}
Each state is based on it's unique collision vector. Also each state has an array of next states, and edge weights in a One-One correspondence with every element of the next states.\\
After some deliberation we came up with the following structures to accommodate the aforementioned specifications:

\begin{verbatim}
struct collisionVector{
	    int arr[COL];
	    int length;
};

struct state{ 
	    int value;
	    struct collisionVector* cv;
	    struct state* next[SIZE];
	    int latency[SIZE];
};
\end{verbatim}

\subsection{Creating a Directed Graph representing the State Diagram}
The function $populateDiagram()$ basically does a depth first population of the state diagram. It takes each node as it is created, and computes every possible child node (and the edges to these child nodes). Later the function is recursively called for every child node if they have not been populated earlier.\\ \\
While each node is being thus populated, we are simulataneously constructing an Adjacency List representation of the Directed Graph which the State Diagram ultimately represents.\\ \\
\textbf {function $populateDiagram()$}\\
\textbf {Input}: 
\begin{itemize}
    \item $struct$ $state*$ $firstState$\\ This is the very first state created from the initial collision vector. This state serves as the seed for the population process of the State Diagram.
    \item $struct$ $collisionVector*$ $initialVector$ \\ This is the initial collision vector which was created from the input reservation table.
\end{itemize}
\textbf {Output}: None\\
\textbf {Global Variables}: 
\begin{itemize}
    \item $struct$ $state*$ $adjacencyList[]$ \\This is an Adjacency List used to store the directed graph notation of the state diagram.
    \item $struct$ $state*$ $traversedArray[]$ \\This is a hashtable used to implement the Depth First population of the states in the state diagram.
\end{itemize}
\textbf{Algorithm}:
\begin{verbatim}
Begin:

1.  Set traversedArray value for firstState as True.
2.  Traverse over the initial collision vector.
    a.  For every right shifted 0, we have two things - 
    a new state configuration, and a latency associated with the new state.
    
    b.  Update this new state and the associated latency 
    as a child node and edge of the given firstState.
    
    c.  Update this as a node in the adjacency list entry for firstState.
    
3.  Mark firstState as traversed.

4.  For every child of firstState recursively call populateDiagram()
if it is not traversed.

End
\end{verbatim}

\subsection{Finding all Elementary Cycles of a Directed Graph}
This problem is a well known NP-Hard problem (possibly even PSPACE). In our ignorance, at first we tried doing a simple DFS search (modified to fit our needs) on the adjacency list. However the search yielded results which were duplicate, and ignored some crucial cycles in the graph. This lead us to realize that there was more to this particular problem than met the eye, and we started researching on the topic.\\
We came up with quite a few papers published from 1970 (by Tiernan) to 1975 (by Johnson) that gave approximate algorithms to solve the problem. The best algorithm in this case was clearly Johnson's algorithm by his own admission. In Donald B. Johnson's 1975 paper on the same topic he says,
\begin{quote}
    ""There are exactly $$\sum_{i=1}^{n-1} \binom{n}{n-i+1}(n-i)!$$ elementary circuits in a complete directed graph with $n$ vertices. Thus the number of elementary circuits in a directed graph can grow faster with n than the exponential $2^n$. So it is clear that our algorithm, which has a time bound of $O((n + e)(c + 1))$ on any graph with $n$ vertices, $e$ edges and $c$ elementary circuits, is feasible for a substantially larger class of problems than the best algorithms previously known which realize a time bound of $O(n. e(c + 1))$.""
\end{quote} 
This shows that even after reducing the Running Time complexity by an order of $O(e.(c+1))$, we were still left with an exponential time complexity since the number of elementary circuits in the graph could itself be exponential in nature.\\
Also just because Johnson's algorithm was the best in terms of running time complexity, didn't mean it was the most adaptable to our situation. Note that Johnson's algorithm itself is not explained in detail here, as we ended up using our own modified version. We had to code the program \textbf{entirely in ANSI C}, and Johnson's algorithm provided the following challenges:
\begin{itemize}
    \item We needed to implement an algorithm to find and return Strongly Connected Components to the main algorithm. This meant implementing either Tarjan's algorithm or Kosaraju's algorithm and somehow returning the set of Strongly Connected Components to the main algorithm to process.
    \item Johnson's algorithm itself needs complicated datastructures like a blocket set and a blocked map, which are quite tricky to implement in C.
\end{itemize}
We made the following discoveries which led to our own modified algorithm:
\begin{itemize}
    \item The Blocking Mechanism in Johnson's algorithm could be ignored with a penalty induced in the running time, but on the other hand simplifying the process manifold by removing dependency on any external Data Structure except for a stack.
    \item The initial state diagram is always a strongly connected component. And even if resultant state diagrams (after removing vertices) aren't strongly connected components, we just incur an extra penalty of $O(n^2)$ at most for searching through the adjacency list looking for cycles that don't exist.
\end{itemize}
A concise algorithm for our implementation is given as follows:


\begin{verbatim}
Begin main():
 	
For every node in the list of vertices:
    a. Create a new stack.
    b. Call the findCycles() method using this node, and the stack.
    c. Remove the current node from the adjacency list.
    d. Recalculate the Hash Table which associates the collision
    vector values of the nodes with respective vertex numbers.
    e. Free the stack.
End

Begin findCycles(stack, current_node):

1. Push current_node onto the stack.
2. Mark the node as present on stack.
3. For every child of current node u do:
    a. If the child of u is the current_node, its a cycle. Send
    the stack to printCycles().
    b. If child of u is already present in stack then move onto 
    the next child. This is our own implementation of the blocking
    mechanism of Johnson's algorithm.
    c. Otherwise push this child onto the stack and recursively call
    findCycles() on this node.
4. Pop stack.
5. Mark node as absent on stack.
6. Return to the parent node to look for alternate routes.

End.

Begin printCycle(stack):

1. Pop every element from the stack and push it into a temporary stack.
2. Pop every element from the temporary stack:
    a. Print this element.
    b. Reinsert it into original stack.

End



\end{verbatim}

\end{document}
